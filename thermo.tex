\documentclass{article}

% Language setting
% Replace `english' with e.g. `spanish' to change the document language
\usepackage[english]{babel}

% Set page size and margins
% Replace `letterpaper' with`a4paper' for UK/EU standard size
\usepackage[letterpaper,top=2cm,bottom=2cm,left=3cm,right=3cm,marginparwidth=1.75cm]{geometry}

% Useful packages

\usepackage{graphicx, float}
% \usepackage{pgfplots}
\usepackage{amsmath,amssymb}
\usepackage{hyperref}
\usepackage[style=nature]{biblatex}
%\usepackage[a4paper, total={6.27in, 9.69in}]{geometry}
\usepackage{booktabs, multirow} 
\usepackage{soul}
\usepackage{changepage,threeparttable} 
\usepackage{subcaption}

\title{Thermoelectric Power Experiment}
\author{Rahul Narwar 190665 }
\begin{document}

\maketitle

\section{Aim}

To calculate 
\begin{itemize}
    \item thermoelectric power (Q) 
    \item Fermi Energy (Ef) 
    \item Carrier concentration (n)
\end{itemize} of a given ferrite sample.

\section{Theory}

Theoretical model of Thermoelectric power suggest emf varies with $\Delta T$ as given in equation 1.

Equation 2 and 3 represent Fermi energy for thermocouples.

\begin{align}
    \mathbf E_\text{emf} = -S \nabla T\\
    QT = E_g-E_f+2KT\\
    QT = E_f-KT
\end{align}

Our sample being ferrite one can ignore the kinetic energy term giving the $E_f$ for n and p type conductors as eq 4,5 respectively-

\begin{align}
    E_f=E_g - QT\\
    E_f= QT
\end{align}

For carrier concentration we have.

\begin{equation}
    Q = - \frac{k_B}{e} \left[ \frac{\ln\beta\left(\frac{N}{n}\right)}{n} + \frac{S_T}{k_B}\right]
\end{equation}
$S_T$= Entropy transport term, which is negligible for ferrite materials. Therefore $S_T/K$ is neglected.\\
$N=$ density of states or number of available sites. \\
$k_B$= Boltzmann constant \\
$e =$ electronic charge
$\beta =$ degeneracy factor which includes both spin and orbital degeneracy and its value is taken as unity ($\beta=1$)
Considering that$n<<N$ we can reduce the above formula to
\begin{equation}
    n = N e^{Qe/k_B}
\end{equation}
\\
V is the volume of the sample and the value of k is found to be 86.4$\mu V/K$, the equation becomes
\\
\begin{equation}
    n = \frac{N}{V}e^{Q/86.4}
\end{equation}
where Q is in $\mu V/K$

\section{Method}

\subsection*{Apparatus Used}
\begin{enumerate}
    \item Thermoelectric power setup
    \item Microvoltmeter
    \item Thermocouple
    \item Temperature indicator
    \item Variac
\end{enumerate}

\subsection*{Software used}
\begin{enumerate}
    \item FFMPEG
    \item ImageJ
    \item MATLAB,Python
\end{enumerate}

\begin{itemize}
    \item Measure T difference
    \item plot emf (mV) vs Tdiff to get best fit line.
    \item Slope of this line is Q.
    \item Proceed with analysis of carrier concentration with this slope Q.
\end{itemize}

\section{Data}

\includegraphics[width=0.8\textwidth]{Plottdiffemf.png}

\begin{table}[]
    \centering
    \begin{tabular}{|p{3cm}|p{3cm}||p{3cm}|p{3cm}|}
    \hline
    T(hot) & T(cold) & T(diff) & emf. \\
    \hline
250 & 51 & 199 & 46.9 \\
240 & 51 & 189 & 45 \\
230 & 51 & 179 & 41.4 \\
223 & 52 & 171 & 39.4 \\
210 & 53 & 157 & 36.4 \\
201 & 53 & 148 & 34.3 \\
195 & 52 & 143 & 33 \\
190 & 52 & 138 & 32.1 \\
184 & 52 & 132 & 30.6 \\
176 & 52 & 124 & 28.5 \\
168 & 52 & 116 & 26.4 \\
159 & 52 & 107 & 24.3 \\
148 & 52 & 96 & 21.7 \\
139 & 52 & 87 & 19.7 \\
128 & 51 & 77 & 17.2 \\
118 & 51 & 67 & 15 \\
108 & 50 & 58 & 13 \\
99 & 48 & 51 & 11.1 \\
91 & 48 & 43 & 9.5 \\
82 & 47 & 35 & 7.8 \\
\hline
    \end{tabular}
    \caption{Data obtained}
    \label{tab:my_label}
\end{table}



\section{Analysis}

\subsubsection{Thermoelectric power co-efficient}

\includegraphics[width=0.8\textwidth]{Qslope.png}
\\
Linear fit of emf vs tdiff.
\\
Slope =      0.2389  (0.2358, 0.2419)
\\
Constant =  -1.008  (-1.389, -0.6278)

\subsection{Fermi Energy}

For p type semiconductors
\\
$E_f = QT$

Mean of readings is 105.14 mV

\subsection{Carrier Concentration}

Volume of element=300$mm^3 = 0.3 cm^3$
\\
Carrier concentration $\eta = \frac{10^{22}}{0.3} \times e^{\frac{0.239}{86.4e-3}} = 5.293 \times 10^{23} cm^{-3}$

\section{Error}

We measure error in slope of graph with 95\% confidence bounds as- 0.0061/2 = 0.003(rounded)
\\
Error in $E_f= 0.761 mV$
\\
\\
Error in $\eta$ is calculated by setting the maximum and minimum possible values of Q(slope), ignoring error in measurment of volume since errors in dimensions of element are unavailable.
\\
%$\Delta \eta = \eta (\frac{\Delta N}{N} + \frac{\Delta V}{V} + \frac{\Delta Q}{86.4})$
\\
$\Delta \eta = \frac{\eta_{max} - \eta_{min}}{2} = 0.182 \times 10^{23} $

\section{Results}

 $Q =  0.239\pm 0.003$ mV/K
 \\
 $E_f=105.14 \pm 0.761 mV$
\\
$\eta=5.293 \pm 0.182 \times 10^{23} cm^{-3}$
\section{Summary}

\subsection{Difficulties}

\begin{itemize}
    \item Several different things are named "Fermi Energy" and it was quite confusing.
    \item The semicondutor Fermi energy was difficult to understand.
\end{itemize}

\subsection{Precautions}

\begin{itemize}
\item Increase in temperature must be slow.
\item See that the temperature should remain constant at least for few minutes, while taking the
readings.
\end{itemize}


\section{Acknowledgements}

I thank our professors,TA's and lab staff for helping us provide data and analysing the experiments.

\section{Q/A}

\begin{itemize}
    \item 1. What do you mean by Seebeck, Peltier and Thomson’s effect?
    \\
    \\
    Seebeck effect is observed due the electromotive force from temperature gradient across a eletrically conducting material.
    \\
    \\
    Peltier effect is associated with behaviour of thermocouple when a electric current is passed through it. Heat evolves at one junction of TC and absorbed at opposite.
    \\
    \\
    Thompson effect is observed in material with varying Seebeck coeffecient with temperature, it descrives the heating or cooling of a current-carrying conductor with termperature gradient.
    \\
    \item 2. What is the basic principle involved in the above three effects?
    \\
    \\
    The co-relation between temperature gradient and migration of electrons in atoms results in observable Voltages across materials, these affects are studied as thermoelectric effects.
    \item 3. What is the difference between photoelectric effect and thermoelectric effect?
    \\
    \\
    Voltage due to excitation of atom electrons by photons on photosensitive materials.
    \\
    \\
    Voltage/electron accln due to application of temperature gradient to thermocouples.
    \item 4. Mention a few applications of thermoelectric power?
    \\
    \\
    The Seebeck effect is used in thermoelectric generators, which function like heat engines but far less efficient in general, combination of heat engines and TE generators however provides increased efficiency. 
    \\
    \\
    The Peltier effect is used in thermal cyclers, laboratory devices used to amplify DNA by the polymerase chain reaction (PCR)
    \\
    \\
    TE effects can be used in electronic Temperature measurement devices.
\end{itemize}




\end{document}
